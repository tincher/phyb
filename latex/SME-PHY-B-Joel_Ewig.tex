\documentclass{article}
\usepackage[english,german]{babel}

\begin{document}

\title{Hausarbeit SME-PHY-B: Wahlthema 1 - Volumimeter}
\author{Joel Ewig}
\date{\today}
\maketitle
\clearpage

\begin{abstract}
Aufgabenstellung
Mein Lösungsansatz
\end{abstract}
\clearpage

\tableofcontents
\clearpage

% kalman: https://www.youtube.com/watch?v=I4DT1jj9ucI
% HMM: https://github.com/OlegZero13/Data-Science-Algorithm-Gallery/tree/master/Hidden%20Markov%20Model
% https://www.mdpi.com/2218-6581/4/1/63

\section{Konzept}
\begin{enumerate}
\item Gyrometer zur Ausrichtung
\item Accelerometer und Gyrometer mittels mehr dimensionalenm Kalman-Filter kombiniert zur Bewegung im Raum
\item beides zusammen in ein HMM zur Erkennung von Aktivität und Zählen
\end{enumerate}
\subsection{Ausrichtung}
% Filter
\subsection{Bewegung im Raum}
% Kalman
\subsection{Prozesserkennung}

\section{Experimente}
\section{Evaluation}




\end{document}
