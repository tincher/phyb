\documentclass{article}
\usepackage[english,german]{babel}

\begin{document}

\title{Hausarbeit SME-PHY-B: Wahlthema 2 - Fitnessz\"ahler}
\author{Joel Ewig}
\date{\today}
\maketitle
\clearpage

\begin{abstract}
Sensorbasiert sollen ähnlich einer Fitnessuhr verschiedene Sportübungen erkennt sowie die Anzahl wie oft diese jeweils ausgeführt wurden.
Benutzt werden sollen der Beschleunigungssensor und das Gyroskop des IMU-6050.\\
Mein L\"osungsansatz gliedert sich in mehrere Stufen:
die Rohdaten aus dem IMU-6050 sollten durch einen mehrdimensionalen Kalman-Filter bereinigt werden.
Die bereinigten Daten werden mit dem K-Means Algorithmus geclustert um daraus sogenannte \glqq{}Beobachtungen\grqq{} zu machen.
In der Lernphase wird mit diesen Beobachtungen pro \"Ubung ein Hidden Markov Model erlernt.
In der folgenden Detektionsphase werden die empfangenen Daten ebenfalls geclustert und die Beobachtungen an alle Hidden Markov Modelle weitergegeben.
Das HMM welches die höchste Wahrscheinlichkeit für die Emittierung der beobachteten Folge bestimmt, gilt als Erkenner und die zugehörige Übung wird als aussgeführt behandelt.
\end{abstract}
\clearpage

\tableofcontents
\clearpage

% kalman: https://www.youtube.com/watch?v=I4DT1jj9ucI
% HMM: https://github.com/OlegZero13/Data-Science-Algorithm-Gallery/tree/master/Hidden%20Markov%20Model
% https://www.mdpi.com/2218-6581/4/1/63

\section{Konzept}
\begin{enumerate}
\item Accelerometer und Gyrometer mittels mehrdimensionalenm Kalman-Filter bereinigt
\item gefilterte Daten werden mittels kmeans geclustert
\item clusterzugeh\"origkeit als beobachtung in ein HMM zur Erkennung von Aktivit\"at und Z\"ahlen
\end{enumerate}
\subsection{Arduino}
Auf dem Arduino werden die Daten des Beschleunigungssensors und des Gyroskop aus dem IMU ausgelesen.
Das Gyroskop sowie der Beschleunigungssensor werden dem im Datenblatt beschriebenem Selbsttest unterzogen.
Dieser soll feststellen, dass die Sensoren noch funktionsf\"ahig sind.
Die zugeh\"origen Daten werden \"uber die serielle Schnittstelle an ein Pythonskript \"ubermittelt, welches die zugeh\"origen Rechnungen vornimmt und bei Nichtbestehen des Selbsttests das Programm beendet.\\

\subsection{Prozesserkennung}

\section{Experimente}
\section{Evaluation}




\end{document}
